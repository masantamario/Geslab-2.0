\documentclass[a4paper,12pt,twoside]{memoir}

% Castellano
\usepackage[spanish,es-tabla]{babel}
\selectlanguage{spanish}
\usepackage[utf8]{inputenc}
\usepackage[T1]{fontenc}
\usepackage{lmodern} % Scalable font
\usepackage{microtype}
\usepackage{placeins}

\RequirePackage{booktabs}
\RequirePackage[table]{xcolor}
\RequirePackage{xtab}
\RequirePackage{multirow}

% Links
\PassOptionsToPackage{hyphens}{url}\usepackage[colorlinks]{hyperref}
\hypersetup{
	allcolors = {red}
}

% Ecuaciones
\usepackage{amsmath}

% Rutas de fichero / paquete
\newcommand{\ruta}[1]{{\sffamily #1}}

% Párrafos
\nonzeroparskip

% Huérfanas y viudas
\widowpenalty100000
\clubpenalty100000

% Imagenes
\usepackage{graphicx}
\newcommand{\imagen}[2]{
	\begin{figure}[!h]
		\centering
		\includegraphics[width=0.9\textwidth]{#1}
		\caption{#2}\label{fig:#1}
	\end{figure}
	\FloatBarrier
}

\newcommand{\imagenflotante}[2]{
	\begin{figure}%[!h]
		\centering
		\includegraphics[width=0.9\textwidth]{#1}
		\caption{#2}\label{fig:#1}
	\end{figure}
}



% El comando \figura nos permite insertar figuras comodamente, y utilizando
% siempre el mismo formato. Los parametros son:
% 1 -> Porcentaje del ancho de página que ocupará la figura (de 0 a 1)
% 2 --> Fichero de la imagen
% 3 --> Texto a pie de imagen
% 4 --> Etiqueta (label) para referencias
% 5 --> Opciones que queramos pasarle al \includegraphics
% 6 --> Opciones de posicionamiento a pasarle a \begin{figure}
\newcommand{\figuraConPosicion}[6]{%
  \setlength{\anchoFloat}{#1\textwidth}%
  \addtolength{\anchoFloat}{-4\fboxsep}%
  \setlength{\anchoFigura}{\anchoFloat}%
  \begin{figure}[#6]
    \begin{center}%
      \Ovalbox{%
        \begin{minipage}{\anchoFloat}%
          \begin{center}%
            \includegraphics[width=\anchoFigura,#5]{#2}%
            \caption{#3}%
            \label{#4}%
          \end{center}%
        \end{minipage}
      }%
    \end{center}%
  \end{figure}%
}

%
% Comando para incluir imágenes en formato apaisado (sin marco).
\newcommand{\figuraApaisadaSinMarco}[5]{%
  \begin{figure}%
    \begin{center}%
    \includegraphics[angle=90,height=#1\textheight,#5]{#2}%
    \caption{#3}%
    \label{#4}%
    \end{center}%
  \end{figure}%
}
% Para las tablas
\newcommand{\otoprule}{\midrule [\heavyrulewidth]}
%
% Nuevo comando para tablas pequeñas (menos de una página).
\newcommand{\tablaSmall}[5]{%
 \begin{table}
  \begin{center}
   \rowcolors {2}{gray!35}{}
   \begin{tabular}{#2}
    \toprule
    #4
    \otoprule
    #5
    \bottomrule
   \end{tabular}
   \caption{#1}
   \label{tabla:#3}
  \end{center}
 \end{table}
}

%
% Nuevo comando para tablas pequeñas (menos de una página).
\newcommand{\tablaSmallSinColores}[5]{%
 \begin{table}[H]
  \begin{center}
   \begin{tabular}{#2}
    \toprule
    #4
    \otoprule
    #5
    \bottomrule
   \end{tabular}
   \caption{#1}
   \label{tabla:#3}
  \end{center}
 \end{table}
}

\newcommand{\tablaApaisadaSmall}[5]{%
\begin{landscape}
  \begin{table}
   \begin{center}
    \rowcolors {2}{gray!35}{}
    \begin{tabular}{#2}
     \toprule
     #4
     \otoprule
     #5
     \bottomrule
    \end{tabular}
    \caption{#1}
    \label{tabla:#3}
   \end{center}
  \end{table}
\end{landscape}
}

%
% Nuevo comando para tablas grandes con cabecera y filas alternas coloreadas en gris.
\newcommand{\tabla}[6]{%
  \begin{center}
    \tablefirsthead{
      \toprule
      #5
      \otoprule
    }
    \tablehead{
      \multicolumn{#3}{l}{\small\sl continúa desde la página anterior}\\
      \toprule
      #5
      \otoprule
    }
    \tabletail{
      \hline
      \multicolumn{#3}{r}{\small\sl continúa en la página siguiente}\\
    }
    \tablelasttail{
      \hline
    }
    \bottomcaption{#1}
    \rowcolors {2}{gray!35}{}
    \begin{xtabular}{#2}
      #6
      \bottomrule
    \end{xtabular}
    \label{tabla:#4}
  \end{center}
}

%
% Nuevo comando para tablas grandes con cabecera.
\newcommand{\tablaSinColores}[6]{%
  \begin{center}
    \tablefirsthead{
      \toprule
      #5
      \otoprule
    }
    \tablehead{
      \multicolumn{#3}{l}{\small\sl continúa desde la página anterior}\\
      \toprule
      #5
      \otoprule
    }
    \tabletail{
      \hline
      \multicolumn{#3}{r}{\small\sl continúa en la página siguiente}\\
    }
    \tablelasttail{
      \hline
    }
    \bottomcaption{#1}
    \begin{xtabular}{#2}
      #6
      \bottomrule
    \end{xtabular}
    \label{tabla:#4}
  \end{center}
}

%
% Nuevo comando para tablas grandes sin cabecera.
\newcommand{\tablaSinCabecera}[5]{%
  \begin{center}
    \tablefirsthead{
      \toprule
    }
    \tablehead{
      \multicolumn{#3}{l}{\small\sl continúa desde la página anterior}\\
      \hline
    }
    \tabletail{
      \hline
      \multicolumn{#3}{r}{\small\sl continúa en la página siguiente}\\
    }
    \tablelasttail{
      \hline
    }
    \bottomcaption{#1}
  \begin{xtabular}{#2}
    #5
   \bottomrule
  \end{xtabular}
  \label{tabla:#4}
  \end{center}
}



\definecolor{cgoLight}{HTML}{EEEEEE}
\definecolor{cgoExtralight}{HTML}{FFFFFF}

%
% Nuevo comando para tablas grandes sin cabecera.
\newcommand{\tablaSinCabeceraConBandas}[5]{%
  \begin{center}
    \tablefirsthead{
      \toprule
    }
    \tablehead{
      \multicolumn{#3}{l}{\small\sl continúa desde la página anterior}\\
      \hline
    }
    \tabletail{
      \hline
      \multicolumn{#3}{r}{\small\sl continúa en la página siguiente}\\
    }
    \tablelasttail{
      \hline
    }
    \bottomcaption{#1}
    \rowcolors[]{1}{cgoExtralight}{cgoLight}

  \begin{xtabular}{#2}
    #5
   \bottomrule
  \end{xtabular}
  \label{tabla:#4}
  \end{center}
}


















\graphicspath{ {./img/} }

% Capítulos
\chapterstyle{bianchi}
\newcommand{\capitulo}[2]{
	\setcounter{chapter}{#1}
	\setcounter{section}{0}
	\chapter*{#2}
	\addcontentsline{toc}{chapter}{#2}
	\markboth{#2}{#2}
}

% Apéndices
\renewcommand{\appendixname}{Apéndice}
\renewcommand*\cftappendixname{\appendixname}

\newcommand{\apendice}[1]{
	%\renewcommand{\thechapter}{A}
	\chapter{#1}
}

\renewcommand*\cftappendixname{\appendixname\ }

% Formato de portada
\makeatletter
\usepackage{xcolor}
\newcommand{\tutor}[1]{\def\@tutor{#1}}
\newcommand{\course}[1]{\def\@course{#1}}
\definecolor{cpardoBox}{HTML}{E6E6FF}
\def\maketitle{
  \null
  \thispagestyle{empty}
  % Cabecera ----------------
\noindent\includegraphics[width=\textwidth]{cabecera}\vspace{1cm}%
  \vfill
  % Título proyecto y escudo informática ----------------
  \colorbox{cpardoBox}{%
    \begin{minipage}{.8\textwidth}
      \vspace{.5cm}\Large
      \begin{center}
      \textbf{TFG del Grado en Ingeniería Informática}\vspace{.6cm}\\
      \textbf{\LARGE\@title{}}
      \end{center}
      \vspace{.2cm}
    \end{minipage}

  }%
  \hfill\begin{minipage}{.20\textwidth}
    \includegraphics[width=\textwidth]{escudoInfor}
  \end{minipage}
  \vfill
  % Datos de alumno, curso y tutores ------------------
  \begin{center}%
  {%
    \noindent\LARGE
    Presentado por \@author{}\\ 
    en Universidad de Burgos\\
    \@date{}\\
    Tutor: \@tutor{}\\
  }%
  \end{center}%
  \null
  \cleardoublepage
  }
\makeatother

\newcommand{\nombre}{Mario Santamaría Arias} %%% cambio de comando

% Datos de portada
\title{Geslab 2.0}
\author{\nombre}
\tutor{Pedro Renedo Fernandez}
\date{\today}

\begin{document}

\maketitle


\newpage\null\thispagestyle{empty}\newpage


%%%%%%%%%%%%%%%%%%%%%%%%%%%%%%%%%%%%%%%%%%%%%%%%%%%%%%%%%%%%%%%%%%%%%%%%%%%%%%%%%%%%%%%%
\thispagestyle{empty}


\noindent\includegraphics[width=\textwidth]{cabecera}\vspace{1cm}

\noindent D. Pedro Renedo, profesor del departamento de Ingeniería Informática, área de Lenguajes y sistemas informáticos.

\noindent Expone:

\noindent Que el alumno D. \nombre, con DNI 71298543S, ha realizado el Trabajo final de Grado en Ingeniería Informática titulado Geslab 2.0. 

\noindent Y que dicho trabajo ha sido realizado por el alumno bajo la dirección del que suscribe, en virtud de lo cual se autoriza su presentación y defensa.

\begin{center} %\large
En Burgos, {\large \today}
\end{center}

\vfill\vfill\vfill

%% Author and supervisor
%\begin{minipage}{0.45\textwidth}
%\begin{flushleft} %\large
%Vº. Bº. del Tutor:\\[2cm]
%D. nombre tutor
%\end{flushleft}
%\end{minipage}
%\hfill
%\begin{minipage}{0.45\textwidth}
%\begin{flushleft} %\large
%Vº. Bº. del co-tutor:\\[2cm]
%D. nombre co-tutor
%\end{flushleft}
%\end{minipage}
%\hfill

\vfill

% para casos con solo un tutor comentar lo anterior
% y descomentar lo siguiente
Vº. Bº. del Tutor:\\[2cm]
D. nombre tutor


\newpage\null\thispagestyle{empty}\newpage




\frontmatter

% Abstract en castellano
\renewcommand*\abstractname{Resumen}
\begin{abstract}
Actualmente en la Facultad de Ciencias de la Universidad de Burgos, existen muchos laboratorios que disponen de existencias de reactivos y productos químicos en sus instalaciones. Es importante que haya una correcta organización para facilitar el hecho de compartir existencias entre laboratorios.

La tarea de gestionar las existencias la tienen los propios usuarios de cada laboratorio con la ayuda de GesLab, una aplicación de gestión de inventario desarrollada en 1999 por Álvaro de Luis de Miguel como proyecto de fin de grado.

Desde que se desarrollo e implementó en la facultad este proyecto ha pasado mucho tiempo, y teniendo en cuenta el avance de la tecnología, la propia facultad planteó la posibilidad de realizar una nueva versión.

Así que para la actualización de la aplicación, se pasará de una aplicación de escritorio a una versión web preparada para se utilizada desde distintos dispositivos, con nuevas funcionalidades y que permita un acceso ágil a todos los usuarios de los laboratorios.
\end{abstract}

\renewcommand*\abstractname{Descriptores}
\begin{abstract}
Gestión de inventario, web app, bases de datos, servidor web, diseño web
\end{abstract}

\clearpage

% Abstract en inglés
\renewcommand*\abstractname{Abstract}
\begin{abstract}
At present, in the Faculty of Science of the University of Burgos, there are many laboratories that have stocks of reagents and chemicals in their facilities. It is important that there is proper organization to facilitate the sharing of stocks between laboratories.

The task of managing the stockpiles is carried out by the users of each laboratory with the help of GesLab, an inventory management application developed in 1999 by Álvaro de Luis de Miguel as an end-of-degree project.

Since this project was developed and implemented in the faculty, a lot of time has passed, and taking into account the advance of technology, the faculty itself raised the possibility of making a new version.

So, in order to update the application, we will move from a desktop application to a web version prepared to be used from different devices, with new functionalities and allowing an agile access to all the users of the laboratories.
\end{abstract}

\renewcommand*\abstractname{Keywords}
\begin{abstract}
Inventory management, web app, database, web server, web design
\end{abstract}

\clearpage

% Indices
\tableofcontents

\clearpage

\listoffigures

\clearpage

\listoftables
\clearpage

\mainmatter
\capitulo{1}{Introducción}

La gestión del inventario para una organización como la Facultad de Ciencias puede ser una tarea de vital importancia. Una mala gestión de los materiales que tiene en sus laboratorios, puede suponer una pérdida de tiempo y de dinero por parte de la Universidad.

Es importante que todas los departamentos empleen un sistema eficiente de gestión, y que todo el personal sepa utilizar este sistema de una manera correcta. Son los propios miembros de los laboratorios los encargados de registrar tanto las entradas como las salidas de material, por lo que el diseño de la aplicación que van a utilizar tiene peso importante. 

Si el usuario se tiene que enfrentar a una aplicación de gestión poco intuitiva y que requiera de un largo y complicado proceso de aprendizaje, probablemente termine frustrado y finalmente abandonará su uso.

%Según un estudio realizado en 2017, una persona hace uso de unas 9 aplicaciones diarias, (https://techcrunch.com/2017/05/04/report-smartphone-owners-are-using-9-apps-per-day-30-per-month/)

El objetivo principal del proyecto consiste en una una actualización del proyecto de Álvaro de Luis De Miguel \cite{proyecto:Geslab-1.0}, cambiando la aplicación de escritorio que el desarrollo en 1999 por una aplicación web alojada en un servidor de la universidad.
\capitulo{2}{Objetivos del proyecto}

El objetivo principal del proyecto es la renovación de la aplicación ya existente en una aplicación web así como la importación de todos los datos que posee el sistema actual al nuevo. También se han de añadir distintos requisitos que con el uso de la aplicación, los usuarios han marcado como necesarios. Entre ellos se puede destacar un sistema que permita almacenar las medidas de seguridad necesarias para la conservación de los productos, así como un sistema capaz de unificar todos los posibles nombres que puede tener un producto, y evitar así problemas de elementos duplicados.

Antes de analizar en profundidad los nuevos requisitos, es importante explicar el funcionamiento actual de la aplicación.


\section{Funcionamiento actual de la aplicación}

Geslab 1.0 permite a un usuario creado por el administrador de la base de datos logearse y acceder a la información que se encuentra en la misma.

De esta manera el usuario puede consultar las existencias actuales de un producto, que se almacena en forma de \textbf{ficha de producto}. Cada ficha se corresponde con un producto guardado en una ubicación, con una calidad determinada, de una marca en concreto y proporcionado por un distribuidor.

De cada ficha puede haber tanto \textbf{entradas} como \textbf{salidas}, y de estas se guarda la fecha, su caducidad, el nº de lote, las unidades, su capacidad y si es residuo o no. Así, cada ficha puede tener varias entradas y salidas y su stock real se calcula con los datos de estas.

El sistema también almacena una serie de datos importantes para su funcionamiento, dejando estos a disposición del propio usuario para que pueda editarlos y añadir nuevos en caso necesario.   

Hay una tabla con todos los \textbf{productos} de los que puede disponer una ficha, junto a información relevante del producto (formula química, precauciones, etc).

De la misma manera se guardan las distintas \textbf{calidades} que puede tener el producto.

El sistema almacena tanto las \textbf{marcas} como los \textbf{proveedores}, añadiendo también información de contacto.

Por otro lado se almacenan los datos de \textbf{departamentos}, \textbf{áreas} y \textbf{centros}, pero a diferencia de los anteriores estos solo podrán ser editados por un usuario administrador de la base de datos.

Con todos estos datos, el usuario al logearse en la aplicación puede consultar las existencias, pudiendo ver las entradas y las salidas de cada ficha, y puede realizar una búsqueda filtrando en función de los muchos campos de los que disponen las tablas.


\section{Nuevo paradigma}

Antes de continuar con el análisis es importante mencionar como se ha realizado un cambio de paradigma en el uso de los objetos de la base de datos.

En la primera versión de Geslab se trata con una mayor importancia a las tablas de entradas y salidas, teniendo estas muchos campos para después poder calcular el stock de manera dinámica restando las salidas a las entradas. 

En la nueva versión esto cambia, pasando a tener mayor importancia la ficha de producto. Esta se lleva prácticamente todos los atributos que tenían las entradas y salidas además de un campo nuevo para almacenar el stock real.

De esta forma el stock siempre estará actualizado, y solamente se actualizará cada vez que haya una entrada o una salida. 


\section{Nuevas funcionalidades}

La actualización de la aplicación va a recibir nuevas funcionalidades, unas requeridas por el propio planteamiento de la actualización, otras requeridas por parte de los usuarios de la aplicación y otros funcionalidades no funcionales que también merece la pena destacar. 

Antes de nada hay que explicar que todos los requisitos mostrados a continuación son un resumen de los requisitos más destacados, y los que se diferencian de la versión anterior de la aplicación. La relación detallada de todos ellos se puede encontrar en los anexos.

\subsection{Requisitos del proyecto}

\begin{itemize}
\item El proyecto ha de mantener todos los requisitos actuales de la aplicación.

\item También se requiere un sistema que permita almacenar las medidas de seguridad necesarias para la conservación de los productos.

\end{itemize}

\subsection{Requisitos de los usuarios}

\begin{itemize}

\item Tienen que existir varios roles de usuario, \textbf{administrador de la aplicación}, que puede editar las tablas y su funcionamiento, \textbf{gestores de inventario}, que pueden gestionar las fichas de productos, y \textbf{usuarios}, que solo pueden ver la información sin conocer la ubicación de estos.  

\item Se han de añadir los siguientes campos a la tabla producto: pureza, peso molecular, formula desarrollada, Nº CAS, Nº EINECS, Nº EC, pictogramas de seguridad, indicadores de peligro, indicadores de prudencia y hoja de seguridad (PDF)

\item El sistema debe de permitir una búsqueda simplificada y una búsqueda avanzada. La búsqueda avanzada deberá de incluir los campos marca, fórmula, CAS, caducidad, fecha adquisición, nombre y localización. El sistema no deberá distinguir entre mayúsculas y minúsculas.

\item El sistema debe de contemplar los distintos sinónimos que poseen los productos, así como identificar el compuesto tanto por su nombre en español como en inglés.
  
\end{itemize}

\subsection{Requisitos no funcionales}

\begin{itemize}

\item La aplicación se desarrollará en lenguaje java con servlets y jsp y se realizará a través de Eclipse.

\item La base de datos se creará en MySql, y se conectará con Java a través del MySql Connector.

\item El diseño de la aplicación se realizará en HTML5 y se utilizará bootstrap 4 

\end{itemize}


\capitulo{3}{Conceptos teóricos}

En este apartado se hará hincapié en distintos conceptos teóricos que son necesarios para la comprensión del proyecto. Es importante analizar los cambios de tecnología que se van a realizar, para poder comprender correctamente su comportamiento. 

\section{MySQL}

MySQL es un sistema de gestión de base de datos relacionales de código abierto con un modelo cliente servidor. Actualmente está considerada como la base de datos open source más popular del mundo sobre todo para entornos de desarrollo web.

La aplicación actual utiliza como sistema gestor de base de datos el SQL Server 6.5 de Microsoft en inglés, mientras que la  nueva actualización se desarrollará en MySQL. Geslab 1.0 esta desarrollada como una aplicación de escritorio de windows, por lo cual es razonable que se eligiera SQL Server.

Con el salto a aplicación web el cambio a MySQL viene de la mano, ya que MySQL es mucho más sencillo de emparejar con cualquier otro idioma, como en nuestro caso es Java.


\section{JSP}

JavaServer Pages (JSP) es una tecnología que ayuda a los desarrolladores de software a crear páginas web dinámicas basadas en HTML y XML. JSP es similar a PHP, pero usa el lenguaje de programación Java \cite{wiki:JSP}

El motor de las páginas JSP se base en servlets, que son programas destinados a ejecutarse en el lado del servidor de modo que puede ampliar las capacidades del mismo.

Por todo ello, la utilización de JSP y Servlets nos permitirá realizar de una manera muy cómoda todo el diseño de las páginas HTML que utilizará la aplicación, evitándonos el tener que escribir infinitas sentencias println. 

Además esto nos ayuda a diferenciar bien la tarea del diseño gráfico de la aplicación por un lado,y por otro todo el comportamiento de programación.

\section{Maven} 

Maven es una herramienta de software para la gestión y construcción de proyectos Java con un modelo de construcción basado en XML. Maven utiliza un POM para describir el proyecto software a construir, sus dependencias de otros módulos y componentes externos, y el orden de construcción de los elementos. 

Además, el motor incluido en su núcleo puede dinámicamente descargar plugins de un repositorio, el mismo repositorio que provee acceso a muchas versiones de diferentes proyectos Open Source en Java, de Apache y otras organizaciones y desarrolladores. \cite{wiki:maven}

Maven nos ayuda mucho a la hora de crear un proyecto web gracias a su posible implementación con Eclipse. La configuración de las dependencias se realiza de manera muy sencilla, desde la configuración de los servlets, a la configuración de tomcat en el proyecto.

\capitulo{4}{Técnicas y herramientas}

En este apartado se mostraran las diferentes técnicas y herramientas que se han utilizado en el desarrollo del proyecto.

\section{Herramientas de programación}

\subsection{MySQL Workbech}

Según wikipedia, MySQL Workbench es una herramienta visual de diseño de bases de datos que integra desarrollo de software, administración de bases de datos, diseño de bases de datos, gestión y mantenimiento para el sistema de base de datos MySQL \cite{wiki:Workbench}. Usando esta herramienta conseguimos que el trabajar con lenguaje MySQL sea muy llevadero, pudiendo hacer pruebas muy rápidamente con las consultas. Además, gracias a su herramienta de ingeniería inversa, resulta muy sencilla obtener un diagrama entidad-relación de todo el sistema.

\subsection{Eclipse IDE}

Eclipse es una plataforma de software compuesta por un conjunto de herramientas de programación de código abierto multiplataforma para desarrollar lo que el proyecto llama "Aplicaciones de Cliente Enriquecido". Esta plataforma, típicamente ha sido usada para desarrollar entornos de desarrollo integrados (del inglés IDE), como el IDE de Java llamado Java Development Toolkit (JDT) y el compilador (ECJ) que se entrega como parte de Eclipse (y que son usados también para desarrollar el mismo Eclipse) \cite{wiki:Eclipse}.

Con Eclipse podemos crear muy fácilmente proyectos desde plantillas del propio programa, como en nuestro caso será Maven. Además Eclipse es un programa que facilita mucho la tarea de navegar por el código, con funciones de refactorización y herramientas de debug, por lo que es una herramienta indispensable para este proyecto.

\subsection{Brackets}

Para la lectura y edición de los diferentes archivos de los que se compone la aplicación, así como la documentación de la versión anterior, se utilizará Brackets, un editor de texto con licencia abierta que nos permite el trabajo con multitud de lenguajes de programación. Es un programa gratuito, muy ligero y con mucho potencial. 

\section{Herramientas de diseño}

\subsection{Bootstrap 4}

Bootstrap es un framework front-end gratuito que nos ayuda a conseguir un desarrollo web de manera rápida y sencilla. Incluye diseños de plantillas basadas en HTML y CSS para tipografías, forms, botones, tablas, navegación, modales, carruseles de imágenes y muchas muchas otras. Además bootsrap te aporta la habilidad de crear de manera muy sencilla, diseños responsive \cite{Bootstrap}. 

El uso de bootstrap nos ha permitido desarrollar un entorno web atractivo, sin la necesidad de crearlo desde 0. Además gracias a las infinitas bibliotecas extra que se pueden encontrar, se ha podido personalizar al máximo el resultado final.

\subsection{CSS BEM}

BEM (Block, Element, Modifier o Bloque, Elemento, Modificador) es una metodología ágil de desarrollo basada en componentes. Su objetivo es dividir la interfaz de usuario en bloques independientes para crear componentes escalables y reutilizables. Propone un estilo descriptivo para nombrar cada una de las clases a utilizar, permitiendo así crear un estructura de código consistente.

Gracias a BEM se ha conseguido estructurar de una manera consistente todo el CSS, mejorando la reutilización del código y simplificando el resultado final.


\subsection{Animation.js}
Esta librería es una biblioteca de animación JavaScript que funciona con propiedades CSS, transformaciones CSS individuales, SVG o cualquier atributo DOM y objetos JavaScript. Con ella se han creado unas animaciones muy simples para mejorar el apartado estético.

\subsection{Adobe Illustrator}
Para la creación de todos los componentes gráficos de la aplicación se ha utilizado la herramienta Illustrator. Adobe illustrator es un editor de gráficos vectoriales destinada a la creación artística de dibujo e ilustración.

\subsection{Figma}
Para la implementación de los diseños creados con Illustrator en la página web en los primeros diseños de wireframes, se utilizó Figma. Esta aplicación brinda todas las herramientas necesarias para la fase de diseño del proyecto, incluidas las herramientas vectoriales capaces de ilustrar completamente, así como aquellas para la creación de prototipos y la generación de código para el traspaso (hand-off). 

En los diseños definitivos, y tras ganar conocimiento sobre el uso de bootstrap, esta herramienta se dejó de utilizar.

\section{Herramientas de gestión}

\subsection{Skype Empresarial y Teams}

Para la comunicación con el tutor del proyecto se han utilizado dos servicios de mensajería instantánea, con licencia provista por la propia universidad. Los primeros meses de desarrollo se mantuvo comunicación a través de Skype Empresarial, hasta que finalmente se paso a Microsoft Teams.

\subsection{GitHub y GitHub Desktop}

Para el control del desarrollo del proyecto, se utilizó la herramienta GitHub, con el objetivo de tener un control de todos los cambios y versiones que se han hecho en la aplicación. 

Para facilitar la tarea de llevarlo al día, se uso la herramienta GitHub Desktop para windows, lo que permite sincronizar de manera sencilla todos los cambios que se van realizando.

\subsection{Texmaker}

Para la creación de esta misma documentación, se ha usado LaTeX, gracias a las multiples características y posibilidades que permite. Para la edición se ha usado la aplicación Texmaker, que integra muchas herramientas necesarias para desarrollar documentos con LaTeX.

\capitulo{5}{Aspectos relevantes del desarrollo del proyecto}

Si hay algo importante a destacar en el desarrollo del proyecto, es sin duda alguna el como ha afectado la irrupción del \textbf{COVID-19} en nuestro país, y como el posterior estado de alarma hizo que todo el proyecto se estableciera en el limbo durante varias semanas.

\section{COVID-19}

Cuando se estaban estableciendo las bases del proyecto, se realizó una reunión con el decano de la facultad y un técnico de laboratorio, en la cual se trataron diversos temas, entre ellos, la necesidad por nuestra parte de unos requisitos funcionales elegidos por los propios usuarios de la aplicación, por lo cual se concreto otra segunda reunión para poder obtener una respuesta. Con fecha para la nueva reunión ya establecida, se decreto el estado de alarma en el país, que afecto a todos los organismos de la universidad e incapacitó la tarea de reunirse. 

De aquí en adelante, toda la labor de comunicarse con los usuarios se convirtió en una tarea larga y tediosa, durante la cual el alumno no podía hacer otra cosa sino que autoformarse y estudiar la documentación que poseía hasta el momento, que pese a contener mucha información, había ciertos criterios técnicos que solo un miembro del laboratorio podía entender.

\section{Obtención de requisitos funcionales}

Finalmente se pudo obtener un borrador de unos requisitos funcionales, a través de una serie de correos enviados entre los miembros del departamento. En los correos la información era poco específica y caótica, por lo que hizo falta realizar una tarea de análisis para poder extraer unos requisitos funcionales claros. 

Tras esta tarea de análisis e investigación tanto de la aplicación como de conceptos de química utilizados, se extrajeron unas historias de usuario a medio completar, con el objetivo de que o bien el decano, o bien los usuarios, aportaran más información, y sobre todo información más específica. 

Al no recibir respuesta por parte de la universidad (de nuevo por motivo del confinamiento) se decidió investigar los requisitos por cuenta propia y establecer unos requisitos finales para poder ir trabajando sobre ellos, dejando claro que los requisitos finalmente elegidos no eran definitivos y quedaban abiertos a posibles cambios en un futuro.

\section{Obtención del material original}

Por otro lado, a parte de la información requerida de los usuarios, también se necesitaba obtener todos los datos de la aplicación ya existente en su sistema. De la misma manera que ocurría con los requisitos, la comunicación era complicada, pero Pedro estuvo en contacto con ellos para poder obtener las tablas y poder replicarlas en un MySql, y así ir trabajando con ellas en traspasar los datos a las tablas nuevas.

Cuando finalmente recibimos la información del servidor, la recibimos en un formato de imagen de disco proveniente de Ghost Norton en formato VHD. Esto nos dejaba con una copia del disco duro en el que estaban las tablas que necesitábamos.
Para montar la imagen utilizamos Virtual Vox NT4, pero por alguna razón la maquina no arrancaba. Tras muchas pruebas encontramos que la causa era la tarjeta de red, no era compatible. Tras esto se realizó una búsqueda de una compatible. Finalmente se utilizó AMD PC net family. 

Con todo listo se pudo ejecutar la imagen, y a través de admin SQL exportamos los datos que necesitábamos.

\section{Cambios en la base de datos}

Es importante mencionar como la base de datos original ha sufrido una serie de cambios,  aparte de el propio paso de SQL Server a MySQL, por supuesto. Con el análisis del funcionamiento original de la aplicación y de los nuevos requisitos funcionales, se plantearon multitud de cambios en las tablas, que incluyen la creación de nuevas, la eliminación de otras, y el cambio tanto de formato como de ubicación de los campos de las mismas.

\section{Nuevo planteamiento de la UI}

Un apartado muy importante para el proyecto ha consistido en la creación de una interfaz de usuario con una usabilidad muy intuitiva y adaptada a los estándares actuales. El paso de una aplicación creada hace varias décadas con Visual Basic 5.0 (con todas las limitaciones que eso conlleva), a una webapp que se comporte como un sistema actual al que los usuarios puedan estar acostumbrados, no ha sido una tarea sencilla. 

La aplicación ha de trabajar con multitud de datos a la vez, unos más relevantes en unas situaciones y otros en otras, teniendo que encontrar en cada vista un punto medio entre mostrar datos de importancia real y componer una vista atractiva visualmente, evitando que la página se sobrecargue de información.

También es importante destacar, el hecho de como las tecnologías actuales cada vez están más enfocadas a los dispositivos móviles, de forma que es muy importante pensar en su portabilidad entre dispositivos. Al trabajar con bootsrap, se deja abierta una puerta a una posible modificación del diseño que haga que todo el sistema se pueda comportar de la misma manera en cualquier dispositivo.  

\section{Implantación}

La implantación del sistema en un servidor es una parte vital para que los usuarios puedan llegar a utilizar la aplicación. Todo el proceso de implantación estaba previsto en un principio para Junio, se solicito el servidor a la universidad para poder comenzarlo antes de la entrega, pero los trámites en una situación como la que estamos pasando, con todo el problema de la cuarentena y el estado de alarma, terminaron por alargarse más de lo previsto, y finalmente la implantación ha quedado pospuesta hasta Julio de 2020. 
Esta prevista sobre un servidor de la propia Universidad, con una URL de laUniversidad de Burgos y con un certificado de seguril SSL.
\capitulo{6}{Trabajos relacionados}

Con respecto a trabajos relacionados que puedan existir en el mercado no hay mucho que mencionar, ya que actualmente no existe ningún producto comercial que permita el funcionamiento particular que solicita la facultad de ciencias. Es importante destacar que la aplicación no gestiona las ventas de productos, únicamente es una herramienta de consulta de existencias, para el control de facturas y herramientas se lleva a través de otra aplicación de la Universidad.

Por ello la única herramienta que hay que mencionar es la primera versión de la misma.

\subsection{Geslab 1.0}
Lo primero que hay que mencionar, es la primera versión de la misma aplicación, creada por Alvaro Luis de Miguel en 1999 como su trabajo de fin de grado \cite{GeslabV1}. 

Aunque es una aplicación desarrollada hace ya muchos años, muchos de los funcionamientos y de la manera de trabajar han sido trasladados a esta nueva versión, actualizando los conceptos necesarios para poder mejorar su funcionamiento.


\capitulo{7}{Conclusiones y Líneas de trabajo futuras}

\section{Conclusiones}

Tras meses de trabajo y con una cuarentena de por medio, finalmente se ha conseguido obtener un producto funcional, que cumple los requisitos marcados de una manera satisfactoria. 

Espero que este trabajo pueda servir de una ayuda real a todos los miembros de los laboratorios, facilitando un poco su día a día.

A nivel personal, el desarrollo de Geslab 2.0 me ha unido de nuevo a muchos sectores de la ingeniería informática que tenía olvidados. Bases de datos, programación en Java, diseño web, todos son aspectos que he refrescado y aprendido en más profundidad, y que me hacen replantearme mis oportunidades laborales de aquí en adelante.

Pero no solo la parte técnica, el poder llevar a cabo un proyecto desde cero, con todo lo que ello implica, me ha acercado a lo que me voy a poder encontrar en cualquier empresa que desarrolle productos de software.

\section{Líneas de trabajo futuras}

\subsection{Herramienta importar y exportar datos} 
Actualmente, los datos de productos no solo se encuentran en la base de datos de la aplicación, los usuarios disponen de sus propios documentos en los que han ido guardando la información. Por ello, una herramienta que pueda facilitar la tarea tanto de importar como de exportar datos sería muy útil en la realidad.

\subsection{Herramienta exportar etiquetas}
Los productos del laboratorio en ocasiones requieren de ser acompañados por una etiqueta de seguridad. Toda esta información esta almacenada en la aplicación, por lo que una herramienta para poder imprimir las etiquetas con un mismo formato, y pudiendo editar los valores sería de gran valor. 

\subsection{Versión móvil}
Hoy día la tecnología está cada vez más enfocada a los dispositivos móviles, por lo que hay que pensar en una forma de adecuarlo a estos tipos de dispositivos. El aspecto gráfico de la aplicación esta creado con bootstrap, lo cual permite crear un diseño responsive. Actualmente solo esta optimizado para dispositivos de escritorio, pero queda la opción de actualizarlo para que sea responsive, o por otro lado, también se puede implementar una API para convertirlo en una aplicación móvil. 


\section{Agradecimientos}

Por último, me gustaría dar las gracias a todos los que me han ayudado en el desarrollo de Geslab 2.0, a los miembros del laboratorio, a Gonzalo, decano de la facultad, a Pedro mi tutor, por lucharlo cada semana junto a mí, a mi familia y a mi pareja por aguantar mi convivencia en unos meses de tanto estrés. 

Gracias.


\bibliographystyle{plain}
\bibliography{bibliografia}

\end{document}
